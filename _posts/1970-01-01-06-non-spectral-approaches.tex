%{
\begin{matlab}
  %}
  % Comment/MATLAB set up code
  importTool('dimred')
  dimredToolboxes
  randn('seed', 1e6)
  rand('seed', 1e6)
  if ~isoctave
    colordef white
  end
  % Text width in cm.
  textWidth = 10
  %{
  %   Start of Comment/MATLAB brackets
\end{matlab}

\chapter{Non-Spectral Approaches}
\label{chap:iterative}

\fixme{Emphasize the fact that Classical MDS and spectral methods are two stage processes: one is the choice of distance measure and one is the choice of visualization of the distances. These iterative methods are alternatives for visualization of the distances. Also include t-SNE here.}

\section{Iterative Methods}
\begin{itemize}
\item Multidimensional Scaling (MDS) 

  \begin{itemize}
  \item Iterative optimisation of a stress function \cite{Kruskal:mds64}.
  \end{itemize}
\item Sammon Mappings \cite{Sammon:nonlinear69}.

  \begin{itemize}
  \item Strictly speaking not a mapping --- similar to iterative MDS.
  \end{itemize}
\item NeuroScale \cite{Lowe:neuroscale96}

  \begin{itemize}
  \item Augmentation of iterative MDS methods with a mapping.
  \end{itemize}
\end{itemize}

\section{Distance Preservation}

\textbf{Local Distance Preservation} 
\begin{itemize}
\item The dimensionality reduction techniques we have discussed so far preserve
  local distances in data space.
\item We'll think of that as a mapping
\end{itemize}
Distance Preservation

\textbf{Forward Mapping}
\begin{itemize}
\item Mapping from 1-D latent space to 2-D data space.\[
  \dataScalar{1}=\latentScalar^{2}-0.5,\,\,\,\,\dataScalar{2}=-\latentScalar^{2}+0.5\]

\end{itemize}
\textbf{Backward Mapping}
\begin{itemize}
\item Mapping from 2-D data space to 1-D latent.\[
  x=0.5\left(\dataScalar{1}^{2}+\dataScalar{2}^{2}+1\right)\]

\end{itemize}
% 
\begin{figure}
  \subfigure[]{

    \includegraphics[width=0.45\textwidth]{../../../fgplvm/tex/diagrams/demBackMapping3}}\hfill{}\subfigure[]{

    \includegraphics[width=0.45\textwidth]{../../../fgplvm/tex/diagrams/demBackMapping6}}

  \subfigure[]{

    \includegraphics[width=0.45\textwidth]{../../../fgplvm/tex/diagrams/demBackMapping9}}\hfill{}\subfigure[]{

    \includegraphics[width=0.45\textwidth]{../../../fgplvm/tex/diagrams/demBackMapping12}}

  \caption{(a) and (b) show the case where the mapping is from the latent space
    to the data space. Here two points close in data space can be far
    in latent space. In (c) and (d) the mapping is from the data space
    to the latent space. Here two points far in data space can be close
    in latent space. }

\end{figure}


\section{Tangled String}
\begin{itemize}
\item Sometimes local distance preservation in data space is wrong.

  \begin{itemize}
  \item The pink and blue ball should be separated.
  \item But the assumption makes the problem simpler (for spectral methods
    it is convex).
  \end{itemize}

\end{itemize}
\begin{figure}
  \includegraphics[width=0.3\textwidth]{../diagrams/stringInTwoD}\includegraphics[width=0.3\textwidth]{../diagrams/stringInTwoDwithBalls}\includegraphics[width=0.3\textwidth]{../diagrams/simpleStringInTwoDwithBalls}
  \caption{Two points from a string entangled in two dimensions can be close together even if they are not close together in the underlying string.}
\end{figure}
Spectral Approaches

\textbf{Good}
\begin{itemize}
\item Unique optimum.
\end{itemize}
\textbf{But}
\begin{itemize}
\item Non trivial for dealing with missing data.
\item Difficult to extend (\emph{e.g.\ }temporal data) in a principled
  way. 
\end{itemize}
Summary
\begin{itemize}
\item We have motivated the need for non-linear dimensionality reduction.
\item Spectral approaches can achieve this, but they don't lead to probabilistic
  models.
\item We are looking for a probabilistic approach to encoding the mapping.
\item Next we will se how point based representations of the latent space
  can be used to achieve this.
\end{itemize}


\section{SNE, t-SNE}

Can we relate to iterative mds and Sammon mappings.
%}

%%% Local Variables:
%%% TeX-master: "book"
%%% End:
