\chapter{The Gaussian Density}



\section{Improper Prior on Mean}
\label{sec:improperMean}

An information matrix with a null space in the constant eigenvector
can be arrived at through a Bayesian treatment of a mean vector. Here
we consider Bayesian treatment of one column of $\dataMatrix$. Since
our model is independent across columns (features) it is simple to
extend it to the entire matrix.
\begin{align*}
  p(\dataVector|m) = & \frac{1}{(2\pi)^{\frac{\numData}{2}}\det{\Sigma}^{\frac{1}{2}}} \exp \left(-\frac{1}{2}(\dataVector - m\onesVector)^\top \Sigma^{-1} (\dataVector - m\onesVector)\right)\\
  p(m) =& \frac{1}{(2\pi)^{1/2}\tau} \exp\left(-\frac{m^2}{2\tau}\right)\\
  p(\dataVector) = &\frac{1}{(2\pi)^{\frac{\numData}{2}}\det{\Sigma +
    \tau\onesVector\onesVector^\top}^{\frac{1}{2}}}\exp
  \left(-\frac{1}{2}\dataVector\left(\Sigma +
      \tau\onesVector\onesVector^\top\right)^{-1}\dataVector\right)
\end{align*}
The information matrix for this marginal distribution is found through
the matrix inversion lemma,
\begin{align*}
  \left(\Sigma + \tau\onesVector\onesVector^\top\right)^{-1} =
  \Sigma^{-1} -
  \frac{1}{\tau^{-2}+\onesVector\Sigma^{-1}\onesVector}\Sigma^{-1}\onesVector\onesVector^\top\Sigma^{-1}
\end{align*}
which implies that for $\tau\rightarrow \infty$ the information matrix
will have a null space in the constant vector (just as the stiffness
matrix/Laplacian matrix has). Since once this operation is applied we
cannot recover the component of $\Sigma$ that was in the null space,
we correct for it by adding a constant vector component to
$\laplacianMatrix$ with unknown variance: $\kernelMatrix^{-1}=
\laplacianMatrix +
\gamma\numData^{-1}\onesVector\onesVector^\top$ which implies
that (since the eigenvectors of $\laplacianMatrix$ are all orthogonal
to the constant vector) $\kernelMatrix$ will have an eigenvalue
$\gamma$ associated with the eigenvector $\numData^{-1/2}\onesVector$.
% \begin{align*}
% \Sigma = \Sigma\laplacianMatrix\Sigma+\frac{1}{\gamma^{-1}+\onesVector\Sigma^{-1}\onesVector}\onesVector \onesVector^\top
% \end{align*}
% so we have 
% \Sigma = \laplacianMatrix^{-1} - \gamma\onesVector\onesVector\top
% If we have 


%%% Local Variables:
%%% TeX-master: "book"
%%% End:
